\chapter{Literature Review}
\label{sec:literature}

Attention to climate change issues is on the rise both among the public and scholars, and \textcite{Larch2017} also join the growing literature on international trade that studies the impacts of carbon tariffs on trade flows, welfare, and carbon emissions. To this end, they are the first to build on the work of \textcite{copeland2005} to isolate the forces impacting carbon emissions in a multi-sectoral, multi-factor gravity model. Hence, the work of \textcite{Larch2017} is a valuable contribution to the literature, and, in this Section, I first argue the key role of carbon tariffs, then briefly report initial findings on the impacts of implementing a carbon tariff in the real world, and finally explore other methods used in the literature.

%As summarized in Section \ref{sec:theorethical}, they additionally exploit the structural gravity model to examine impacts on welfare and carbon emissions. To this end, they are the first to build on the work of \textcite{copeland2005} to isolate the forces impacting carbon emissions in a multi-sectoral, multi-factor gravity model. 

\section{Why Carbon Tariffs}

The environment and its components, such as the atmosphere or the oceans, are examples of public goods and, therefore, non-excludable and non-rivalrous. While everyone wants to breathe clean air and drink clean water, the production of goods and services is polluting and produces carbon emissions, ultimately negatively affecting public goods consumption.

There is broad consensus among economists that an efficient and effective way to internalize the externality caused by emissions is to implement a (Pigouvian) carbon tax \parencite{Geoffrey2019}. In order to quantify the "marginal cost of externality" caused by an additional unit of emissions, the social cost of carbon (SCC) is estimated and the Pigouvian tax is set equally \parencite{Pindyck2017}. Mechanically, pricing the carbon intensity of a good increases its price, inducing, for example, producers to improve their energy efficiency or switch to cleaner sectors.

Moreover, environmental protection suffers from the problem of free-riding. Namely, after signing an international agreement to tackle carbon emissions, such as the Copenhagen Accord of 2009, countries do not fully commit to their pledges and rely on the emission reductions of others, ultimately leading to a sub-optimal outcome. In this context, carbon leakage is a failure, and in the analysis of \textcite{Larch2017} it reaches a level of $13.4\%$ in the base case without tariffs, and drops to $1.41\%$ when both Annex I and Annex II countries meet their targets and tariffs are in place. These rates are very similar and are further supported by \textcite{King2021} where, when countries act alone, the leakage amounts to $16.1\%$, while, when the majority of countries complies with agreements (in their case the 2015 Paris Agreement), the leakage decreases to about $3\%$. All in all, these outcomes confirm the validity of carbon tariffs as a tool to tackle carbon leakage.

\section{The World's First Carbon Tariff and Climate Club}

Although the broad consensus and the analyses of \textcite{Larch2017} and \textcite{King2021} highlight the ability of carbon tariffs to reduce carbon emissions and carbon leakage, no country in the world had implemented such a system before October 2023.

Indeed, it was only with the European Regulation 2023/956, which came into force on 17 May 2023, that the European Union declared the introduction of a Carbon Border Adjustment Mechanism (CBAM) as of 1 October 2023. Without going into too much detail, the goal of this instrument is to restore the competitiveness of European goods and address carbon leakage by setting a price on carbon emissions from goods entering the European market. As it is designed, it resembles the well-known ‘climate clubs’ proposed by \textcite{Nordhaus2015}, indeed \textcite{Szulecki2022} calls it a "de facto" climate club, analyses their common features, and points out the governance challenges this instrument will have in the future.

Exploiting the model of \textcite{Larch2017} described in the previous section, \textcite{Korpar2023} analyse the impact of the European CBAM on exports, real GDP, welfare, and carbon emissions. Since both base their analysis on the same multi-regional and multi-sectoral structural gravity model, their results are easily comparable. In terms of trade flows, both find a contraction: \textcite{Larch2017} estimate a decrease in aggregate world trade flow of $1.9\%$, while \textcite{Korpar2023} find a decrease in global exports of $0.11\%$. This small negative effect on European exports is explained by the lower demand for imports from non-EU countries, in fact, non-EU countries experience a slight decrease in their real GDP and, consequently, welfare. In contrast, both European real GDP and welfare increase by about $0.02\%$, due to higher domestic production. The same conclusions come from \textcite{Larch2017} where, in terms of welfare and real income, most (developing) countries are found to be negatively impacted by the redistribution of production between states, while countries with high carbon taxes benefit. Lastly, the European CBAM slightly increases European emissions by $0.24\%$ and decreases global emissions by $0.08\%$, confirming its (modest) effectiveness in addressing carbon leakage and reducing emissions globally.

\section{Other Models and Decomposing Carbon Leakage}

So far, we have reviewed the results of papers that have developed their analyses on the basis of a structural gravity model to estimate the impact of a carbon tariff on trade flows. However, a few other frameworks are also used in the international trade literature to achieve this goal, for example, the Dynamic Computable General Equilibrium (DCGE) model, as in \textcite{Zhang2019}, in which the authors study the impact of US carbon tariffs on China's trade flows, carbon emissions, and welfare.

Another particularly interesting example is the work of \textcite{Tan2018}, which uses a multi-regional Computable General Equilibrium (CGE) model to investigate and decompose the channels through which a carbon tax (in their case a regional Emissions Trading System) generates carbon leakage. Their hypothesis is that carbon leakage occurs mainly through three different channels: 1) competitiveness, which causes the relocation of energy-intensive sectors to regions with more permissive policies; 2) energy, the reduction in production lowers energy prices by sparking fossil fuel consumption in other regions; 3) demand, due to income variation. To this end, they use data from the Hubei ETS and find that the trading scheme succeeds in reducing carbon emissions by 6731 kt CO2, but the surrounding regions increase emissions by 892 kt, causing a carbon leakage of 13.25\%. Isolating the forces, it turns out that the competitiveness channel is the main driver of carbon leakage, accounting for 41.98\% of the increase, while energy and demand account for 7.61\% and 6.21\% respectively. To conclude, the authors propose the introduction of a carbon price on imports from Hubei's neighboring regions, as the competitiveness channel is the main source of carbon leakage, effectively supporting the effectiveness of a carbon tariff on imports to tackle carbon leakage.
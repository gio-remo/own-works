\chapter{Conclusion}
\label{sec:conclusion}

\textcite{Larch2017} analyze the impacts of introducing carbon tariffs on trade flows, welfare, and emissions effects. Moreover, building on the work of \textcite{copeland2005}, they decompose the forces influencing emissions into three effects: scale, composition, and technique. To this end, they are the first to implement such analysis by exploiting a structural multi-sector and multi-factor gravity model.

Firstly, to address the externalities of carbon emissions, they estimate the consequences of introducing a pure carbon tariff on imports, that is, a tax equal to the carbon tax differential between each pair of countries. They find: 1) global trade flows decrease by 1.9\%, 2) 79\% of the countries experiences a welfare loss, and developing countries suffer the most, 3) however, world carbon emissions decrease by 0.50\%. Breaking down the forces behind the change in emissions, they reveal that the main driver is the composition effect, which means that low-carbon sectors have increased their share in global production.

Secondly, they estimate once again the impacts of carbon tariffs on trade, welfare, and carbon emissions, in case a subgroup of countries (Annex I from the Kyoto Protocol) fully reaches its climate pledges. The goal is to explain whether carbon tariffs can also reduce carbon leakage, a major problem when it comes to meeting the targets of international agreements. They point out that: 1) in the case where the Annex I group reaches its pledges but carbon tariffs are not in place, carbon leakage amounts to 13.4\%, and emissions decrease by 8.4\%, 2) on the other hand, with carbon tariffs, carbon leakage falls to 4.14\%, and emissions shrink by 9.3\%.
\chapter{Conclusion}
\label{sec-5-conc}

Micro and small enterprises (MSEs) employ thousands of people around the world, particularly in developing countries. However, they are typically characterized by low productivity and their owners lack sufficient knowledge of basic management practices. Business training programs run by international organizations aim to tackle this problem and their design is often examined by researchers. In "The impact of soft-skills training for entrepreneurs in Jamaica", \cite{Ubfal2022} evaluate the effects of two types of training on 945 entrepreneurs in Jamaica. Based on the concept that personal initiative and perseverance are key components of an entrepreneur's success, one group received intensive classes on these topics. After three months, results revealed positive and statistically significant effects of soft skills training on profits, with a standard deviation of 0.11 compared to the mean of the control group. However, while both men and women began to adopt more recommended business practices after soft skills training, only men recorded better business outcomes after three months. While no significant effect was captured for the other treatment (i.e., combined training), the main intermediate mechanism explaining the short-term increase in profits was the adoption of new business practices. When it comes to designing corporate training, these findings highlight the power of including soft-skills in increasing the adoption of business practices among men and women and positively influencing profits. Soft-skills are not the only valid extension; other studies have also shown the effectiveness of using mentors to assist participants after training. In conclusion, further research should focus, for instance, on addressing the gender heterogeneous effects on business outcomes of business training to truly enable a shared, inclusive, and sustainable development.
\chapter{Discussion}
\label{sec-4-discussion}

\section{Strengths and Limitations}

With "The impact of soft-skills training for entrepreneurs in Jamaica" \cite{Ubfal2022} provide evidence that soft-skills training is a valuable tool for increasing the effectiveness of business training programs. Specifically, their study was able to find a strong and positive treatment effect in the short term for promoting personal initiative and entrepreneurial mindset among business owners. Furthermore, while impacts on business outcomes were found only in the first follow-up, thanks to additional data collected during the second follow-up, soft-skills training appeared to enhance the perseverance, and more generally the soft-skills, of participants even 12 months after the training. Another valuable feature of this work is, for instance, the good quality of the data used. In fact, preliminary selection according to certain criteria and subsequent randomization and stratification provided particularly balanced and thus comparable groups. The characteristics within each group are particularly homogeneous, as also confirmed by the orthogonality test in Table \ref{sum-baseline-balance}, which does not reject the null hypothesis and thus makes the analysis even more robust.

Since \cite{Ubfal2022} adopted an RCT design, treatment compliance and study participation (i.e., attrition) are important factors. The authors pointed out that almost all participants adhered to their assigned group, eliminating any problems related to defiers, but the rate of participation in classes and follow-up surveys was not particularly high, as in other studies. In fact, only 73\% of participants responded to the first follow-up and even fewer to the second, 59\%. Such losses in the follow-up survey undermine the validity and statistical power of the study, and to avoid this, the authors performed several robustness tests. Nevertheless, \cite{Ubfal2022} pointed out that participants were reluctant to arrange follow-up meetings or often rescheduled them because of the popularity of scams in Jamaica. The authors were able to reduce this by introducing a reward for participating in the survey but with only weak results. In my opinion, in addition to cash incentives, possible solutions to the high level of dropout could be, for example, keeping contact with participants alive by sending them regular, personalized updates and reminders. These communications, via e-mail or post, could also be used to deliver additional materials. Hopefully, when scheduling follow-up, participants would trust more the e-mail address or mail sender from whom they have received regular and useful messages, effectively creating a trusted channel. Another weakness of their work is the modest sample size. Initially, the contact list of interested entrepreneurs had about 2000 contacts, while the final sample size is only 945. Personally, I do not see this drastic reduction in number as a mere negative aspect, as the selection reduced heterogeneity and ensured robust results. However, as in any experiment, a larger sample size statistically increases the power and validity of the results and, as also suggested by \cite{McKenzie2014}, future studies should strive for a sample size of several thousand or more. One last limitation of the paper under review may be its time horizon. Other studies, such as the already mentioned \cite{Campos2017} and \cite{Bakhtiar2022}, cover 2 and 3 years, respectively, compared with only 12 months for \cite{Ubfal2022}. Because some effects may take longer to occur and thus to be detected, and different treatments may vary in the persistence of their effects, in this case, further follow-up surveys at 18 or 24 months would have helped to either confirm or not the effectiveness of soft-skills training.

\section{New Directions}

A critical issue often challenged with these training programs is their financial sustainability, as participants often do not experience large increases in profits and fixed costs are barely covered. For this reason, improving the efficiency of these programs as much as possible is critical in order to avoid wasting funds. In Chapter \ref{sec-literature-rev}, I have already mentioned some common extensions; here, I examine some potential further improvements and ideas.

The inclusion of mentoring is likely to increase the adoption of business practices and positively influence profits. \cite{Ubfal2022} themselves suggest including individualized follow-up interventions, but this incurs even higher costs. An alternative design of traditional mentoring could reduce costs, for example, by making these meetings online or by phone, or, as in the Ethiopia trial by \cite{Bakhtiar2022}, by designating previous participants as mentors instead of experienced ones. The idea of using inexperienced mentors may seem less effective, but I think that hiring previous participants as mentors might be more beneficial, as they may have experienced similar problems and know more about the intrinsic and cultural characteristics of the market in which they operate.

Another crucial aspect in the design of corporate training is the validity and quality of the materials to include. \cite{Ubfal2022} used both novel materials and standard contents from the International Labor Organization courses. However, although the materials have been adapted to the local context, I believe it can take some time for trainers to get used to and become familiar with the materials, or to test the actual validity of the content. Therefore, since everything depends on what is taught, special attention should be paid to the creation of content, also allowing for periodic updates and corrections. In addition, as also suggested by \cite{McKenzie2020}, content should not only be tailored by country but also by company-specific sectors, since, for example, a mini-market experiences different challenges than a sawmill. The idea is that a tailored training program based on local and firm-specific characteristics is potentially more effective and efficient than a standardized one.

